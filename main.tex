% !TeX root = main.tex
\documentclass[
    ngerman,
    accentcolor=3b,
    % dark_mode, % uncomment for dark mode
    fontsize= 12pt,
    a4paper,
    aspectratio=169,
    colorback=true,
    fancy_row_colors,
    leqno,
    fleqn,
    boxarc=3pt,
    fleqn,
    main,
    design=2008,
    % shell_escape = false, % Kompatibilität mit sharelatex
]{algoslides}
\RequirePackage{import}
\subimport{common}{preamble}

\graphicspath{{pictures}}
\title[\shortworkshoptitle{}]{\workshoptitle{}}
\department{TU Darmstadt | Fachbereich Informatik | \ophase{} | \insertshorttitle{}}
\date{\today}

%%-------------------------%%
%%--Beginn des Dokumentes--%%
%%-------------------------%%

\begin{document}

    %%-----------%%
    %%--Titelei--%%
    %%-----------%%

    \maketitle{}

    \begin{frame}[c]
        % Die Begrüßenden Worte können individuell pro Workshop festgelegt werden. Am Besten nur was kurzes, sowie "Gude", "Hi", "Herzlich Willkommen!", ...
        \centering\huge\textbf{\gruesswoerte{}}
    \end{frame}

    %%---------------------------%%
    %%--Beginn der Präsentation--%%
    %%---------------------------%%

    \begin{frame}
        \frametitle{Das steht heute auf dem Plan}
        \tableofcontents[subsubsectionstyle=hide]
    \end{frame}


    %%-----------%%
    %%--Kapitel--%%
    %%-----------%%

    % !TeX root = chapter01.tex
\documentclass[
    ngerman,
    accentcolor=3b,
    % dark_mode, % uncomment for dark mode
    fontsize= 12pt,
    a4paper,
    aspectratio=169,
    colorback=true,
    fancy_row_colors,
    leqno,
    fleqn,
    boxarc=3pt,
    fleqn,
    main,
    design=2008,
    % shell_escape = false, % Kompatibilität mit sharelatex
]{algoslides}
\RequirePackage{import}
\subimport{../common}{preamble}

%%--------------------------%%
%%--Imports from Main File--%%
%%--------------------------%%

% Get Labels from Main Document using the xr-hyper Package
\externaldocument[ext:]{../main}
% Set Graphics Path, so pictures load correctly
\graphicspath{{../pictures}}

\begin{document}
    \section{Einführung}\label{1}\label{Einfuehrung}
    \subsection{Subsection1}
    \begin{frame}[<+(1)->]
        \slidehead{}
        \begin{itemize}
            \item Item 1
            \item Item 2
            \item Item 3
        \end{itemize}
    \end{frame}

    \subsection{Subsection2}
    \begin{frame}
        \slidehead{}
        More text
    \end{frame}
\end{document}

    % !TeX root = chapter02.tex
\documentclass[
    ngerman,
    accentcolor=3b,
    % dark_mode, % uncomment for dark mode
    fontsize= 12pt,
    a4paper,
    aspectratio=169,
    colorback=true,
    fancy_row_colors,
    leqno,
    fleqn,
    boxarc=3pt,
    fleqn,
    main,
    design=2008,
    % shell_escape = false, % Kompatibilität mit sharelatex
]{algoslides}
\RequirePackage{import}
\subimport{../common}{preamble}

%%--------------------------%%
%%--Imports from Main File--%%
%%--------------------------%%

% Get Labels from Main Document using the xr-hyper Package
\externaldocument[ext:]{../main}
% Set Graphics Path, so pictures load correctly
\graphicspath{{../pictures}}

\begin{document}
    \section{Basics}\label{2}\label{Basics}
    \subsection{Linux}
    \begin{frame}[<+(1)->]
        \slidehead{}
        TODO
    \end{frame}
    \subsection{Git}
    \begin{frame}[<+(1)->]
        \slidehead{}
        TODO
    \end{frame}
    \subsection{Open Source}
    \begin{frame}[<+(1)->]
        \slidehead{}
        TODO
    \end{frame}
    \subsection{Was ist ein NAS?}
    \begin{frame}[<+(1)->]
        \slidehead{}
        TODO
    \end{frame}
\end{document}

    % !TeX root = chapter02.tex
\documentclass[
    ngerman,
    accentcolor=3b,
    % dark_mode, % uncomment for dark mode
    fontsize= 12pt,
    a4paper,
    aspectratio=169,
    colorback=true,
    fancy_row_colors,
    leqno,
    fleqn,
    boxarc=3pt,
    fleqn,
    main,
    design=2008,
    % shell_escape = false, % Kompatibilität mit sharelatex
]{algoslides}
\RequirePackage{import}
\subimport{../common}{preamble}

%%--------------------------%%
%%--Imports from Main File--%%
%%--------------------------%%

% Get Labels from Main Document using the xr-hyper Package
\externaldocument[ext:]{../main}
% Set Graphics Path, so pictures load correctly
\graphicspath{{../pictures}}

\begin{document}
    \section{Demonstration}\label{2}\label{Demonstration}
    \begin{frame}[<+(1)->]
        \slidehead{}
        \boldimpact{Demonstration}
    \end{frame}
\end{document}

    % !TeX root = chapter02.tex
\documentclass[
    ngerman,
    accentcolor=3b,
    % dark_mode, % uncomment for dark mode
    fontsize= 12pt,
    a4paper,
    aspectratio=169,
    colorback=true,
    fancy_row_colors,
    leqno,
    fleqn,
    boxarc=3pt,
    fleqn,
    main,
    design=2008,
    % shell_escape = false, % Kompatibilität mit sharelatex
]{algoslides}
\RequirePackage{import}
\subimport{../common}{preamble}

%%--------------------------%%
%%--Imports from Main File--%%
%%--------------------------%%

% Get Labels from Main Document using the xr-hyper Package
\externaldocument[ext:]{../main}
% Set Graphics Path, so pictures load correctly
\graphicspath{{../pictures}}

\begin{document}
    \section{Fazit}\label{2}\label{Fazit}
    \subsection{Ist das was für mich?}
    \begin{frame}
        \slidehead{}
        \begin{itemize}
            \item \textbf{Ja}, wenn du:\begin{itemize}
                    \item Interesse an Technik hast
                    \item gerne Probleme löst
                    \item dir Datenschutz wichtig ist
                    \item du dich von Abos entkoppeln willst
                    \item \textbf{genug} Zeit investieren kannst
                    \item \textbf{genug} Geduld hast
                \end{itemize}
            \item \textbf{Nein}, wenn du:\begin{itemize}
                    \item kwillst, dass immer alles funktioniert
                    \item deine Zeit lieber anders investierst
                    \item Informatik nur wegen des Geldes studierst
                \end{itemize}
        \end{itemize}
    \end{frame}

    \subsection{Wie kann ich mich weiter informieren?}
    \begin{frame}
        \slidehead{}
        \begin{itemize}
            \item Youtube
            \item Reddit
                \begin{itemize}
                    \item \href{https://www.reddit.com/r/homelab/}{r/homelab}
                    \item \href{https://www.reddit.com/r/selfhosted/}{r/selfhosted}
                    \item \href{https://www.reddit.com/r/opensource/}{r/smarthome}
                    \item \href{https://www.reddit.com/r/homelab/}{r/homedatacenter}
                    \item \dots
                \end{itemize}
            \item Doku lesen
            \item Fachliteratur
        \end{itemize}
    \end{frame}

    \subsection{Was haben wir uns angeschaut? (Quiz)}
    \begin{frame}
        \slidehead{}
        \begin{itemize}
            \item Was ist Linux?
            \item Was ist Open Source?
            \item Was ist ein NAS?
            \item Muss ich meine Niere verkaufen, um mir einen Server zu leisten?
        \end{itemize}
    \end{frame}
\end{document}


    \section{Verabschiedung}
    \begin{frame}[c]
        \frametitle{Verabschiedung}
        \centering
        \huge\textbf{Vielen Dank für Eure Aufmerksamkeit!}
    \end{frame}

\end{document}
